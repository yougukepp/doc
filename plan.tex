%\documentclass[12pt,a4paper]{book}
\documentclass[12pt,a4paper]{report}

\usepackage{xeCJK}
\usepackage{makeidx}

%楷体
\setCJKmainfont{AR PL UKai CN}

% 将日期变为中文格式
\renewcommand{\today}{\number\year 年 \number\month 月 \number\day 日}

\makeindex
\printindex
\setcounter{secnumdepth}{5}

\title{路由器反向供电商业计划书}
\author{微度无限}

\begin{document}
\maketitle
\tableofcontents

%最多三级标题
%\chapter{章}
%\section{节}
%\subsection{子节}

%%本命令用于重新定义摘要标题为中文"摘要"
\renewcommand{\abstractname}{摘\quad 要}
\begin{abstract}
宽带运营商的机房设备(交换机等)供电均来自与小区供电系统。受小区物业管理、电费等因素影响,运营商在机房投入高昂。本项目可以大幅度减小运营商的机房成本。
\end{abstract}

\chapter{团队介绍}
待施工……

陈卫平

卫利君

刘云

邱锦富

魏晓东

张丹

樊荣

彭鹏

\chapter{需求及解决方案}
本章描述假定交换机工作电源为9V直流。由于目前宽带运营商机房供电基本采取独立供电方案,先介绍这钟方案并分析其弊端:
\section{独立电源供电}
该方案如图\ref{独立电源供电}:
\begin{figure}
    \begin{center}
    \includegraphics[width=0.5\textwidth]{fig/传统供电.jpg}
    \caption{独立电源供电}
    \label{独立电源供电}
    \end{center}
\end{figure}

使用独立电源供电,交换机置于小区机房内,交换机电源适配器从机房220V交流电源取电,转换为9V直流电源(假设交换机工作于9V直流)后供给交换机,交换机与家庭路由器的通信链路只传输数据信号。
这种供电方式是目前大多数运营商采用的供电方式,这种方式有两个弊端:
\begin{enumerate}
    \item 电费按照商业用电标准计费,费用高昂。
    \item 供电收到小区物业制约,运营权利受限。
\end{enumerate}
\section{路由器反向供电}
该方案如图\ref{路由器反向供电}:
\begin{figure}
    \begin{center}
    \includegraphics[width=0.5\textwidth]{fig/反向供电.jpg}
    \caption{路由器反向供电}
    \label{路由器反向供电}
    \end{center}
\end{figure}

使用路由器反向供电,首先在家庭路由器中加入POE供电功能,然后利用路由器与交换机的通信双绞线内的空闲信号线作为供电电源线,将路由器内的电源(24V/48V)通过通信链路传输至机房交换机端,在机房内配置供电转换盒,转换盒将电源与数据信号进行分离,数据信号按正常信号连接方式连接至交换机,电源按照交换机供电要求转换为9V直流后供给交换机。
该方案解决了独立电源供电方案的两个弊端,但是出现了几个新问题,先分别介绍并给出对策:
\begin{enumerate}
    \item 需要在用户从用户家里取电,需要征得用户同意后方可实现。对于这个问题,我们可以对用户提供相应的资费优惠抵消用户的电费支出。
    \item 该方案在机房端需要供电转换盒,需要相应的成本支出。对于这个问题,市场分析章节将具体介绍。
    \item 传统家庭路由器并未实现POE供电功能。对于这个问题,我们可以促进运营商对于路由供货商提出定制具有反向供电功能的路由器,对于路由器的供应商而言,这种改动非常小,也不会改变路由器的外观及核心功能,成本极小,基本可以忽略不计。
    \item 由于通信电缆线一般裸露在室外,雷击等会造成安全隐患。对于这个问题,可以在路由器和供电转换盒的电源接口处做安全保护。
\end{enumerate}
\chapter{市场分析}
本章结合路由器反向供电方案,首先计算单台交换机的改造成本,然后分析其产生的效益,最后分析全球市场容量。
\section{成本}
采用路由器反向供电方案后,需要对现有路由器、交换机进行改造,改造成本是一次性投入。每台交换机的改造成本为:
\begin{equation}
    cost = cC + nPort \times cRP + cCP
    \label{成本}
\end{equation}
其中各变量的含义如表\ref{成本公式变量}
\begin{table}[!hbp]
    \begin{center}
        \begin{tabular}{|l|l|l|}
            \hline
            变量名 & 涵义 & 值(元) \\
            \hline
            cC & 供电转换盒成本 & 30 \\
            \hline
            nPort & 交换机端口数 & 8 \\
            \hline
            cRP & 路由器供电安全保护本 & 5 \\
            \hline
            cCP & 供电转换盒子安全保护成本 & 5 \\
            \hline
            cost & 单台交换机改造成本 & 75 \\
            \hline
        \end{tabular}
        \caption{成本公式变量\label{成本公式变量}}
    \end{center}
\end{table}
由公式\ref{成本}可以计算出单台交换机样品改造成本为75元人民币,批量生产后成本有望下降1/3,即50元/台以内。
\section{效益}
采用独立电源供电方案,运营商的成本有两个方面构成:
\begin{enumerate}
    \item 派遣人员与物业沟通的费用
    \item 电费
\end{enumerate}
保守计算将第一项费用忽略,这里只计算电费。首先给出独立供电的电费计算公式\ref{独立供电电费}:
\begin{equation}
    costOrig = pS * 24 * 365 * pB
    \label{独立供电电费}
\end{equation}
然后同理可得路由器反向供电电费\ref{路由器反向供电电费}:
\begin{equation}
    costNow = pS * 24 * 365 * pR
    \label{路由器反向供电电费}
\end{equation}
路由器反向供电的效益为独立供电成本减去路由器反向供电成本,得出效益公式\ref{效益}:
\begin{equation}
    benefit = pS * 24 * 365 * (pB - pR)
    \label{效益}
\end{equation}
公式\ref{独立供电电费}公式\ref{路由器反向供电电费}公式\ref{效益}中各变量的含义如表\ref{效益变量}
\begin{table}[!hbp]
    \begin{center}
        \begin{tabular}{|l|l|l|}
            \hline
            变量名 & 涵义 & 值 \\
            \hline
            costOrig & 独立供电成本(元/年) & 124.83 \\
            \hline
            costNow & 反向供电成本(元/年) & 75.29 \\
            \hline
            pS & 交换机功率(W) & 15 \\
            \hline
            pB & 商业电价(元/度) & 0.95 \\
            \hline
            pR & 居民电价(元/度) & 0.573 \\
            \hline
            benefit & 单台效益(元/年) &  49.54 \\
            \hline
        \end{tabular}
        \caption{效益变量\label{效益变量}}
    \end{center}
\end{table}
将表\ref{效益变量}中的值带入公式\ref{效益}:中可以得到单台交换机采用路由器反向供电后每年的效益为49.54元/年,结合之前的改造成本50元/台,故可以得出大概一年就可以收回改造成本的结论。
\section{市场容量}
据宽带论坛(BroadbandForum)最新发布的报告称,2013年全球宽带增长达到了新高,而中国的宽带用户数则位列全球第一位。Broadband Forum的报告显示,去年全球宽带的增长达到了过去5年来的新高,其中全球宽带用户增长6549.36万,宽带用户总数达到了5.97亿,年增长率达到12.3\%。另外,在各个国家和地区中,中国宽带用户数达到1.58亿,位列全球第一,年增长率为20.35\%。根据经验数据,交换机和家用路由器的比例为1:10,故交换机的数量约为用户数的1/10。
由以上报告可以得出:
\begin{enumerate}
    \item 全球市场容量约为:30亿/年
    \item 中国的市场容量为:7.5亿/年
    \item 全球新增效益:3亿/年
    \item 中国新增效益:1亿/年。
\end{enumerate}

\chapter{竞争性分析}
待施工……

\end{document}

